\documentclass[10pt]{article}
\usepackage[utf8]{inputenc}	% Para caracteres en español
\usepackage{amsmath,amsthm,amsfonts,amssymb,amscd}
\usepackage{multirow,booktabs}
\usepackage[table]{xcolor}
\usepackage{fullpage}
\usepackage{lastpage}
\usepackage{enumitem}
\usepackage{fancyhdr}
\usepackage{mathrsfs}
\usepackage{wrapfig}
\usepackage{setspace}
\usepackage{calc}
\usepackage{multicol}
\usepackage{cancel}
\usepackage[retainorgcmds]{IEEEtrantools}
\usepackage[margin=3cm]{geometry}
\usepackage{amsmath}
\newlength{\tabcont}
\setlength{\parindent}{0.0in}
\setlength{\parskip}{0.05in}
\usepackage{empheq}
\usepackage{framed}
\usepackage[most]{tcolorbox}
\usepackage{xcolor}
\usepackage{float}
\usepackage[english]{babel}
\usepackage[utf8]{inputenc}
\usepackage{graphicx}
\usepackage[colorinlistoftodos]{todonotes}
\usepackage[version=4]{mhchem}
\usepackage{physics}
\usepackage{mathtools}
%\DeclarePairedDelimiter\bra{\langle}{\rvert}
%\DeclarePairedDelimiter\ket{\lvert}{\rangle}
%\DeclarePairedDelimiterX\braket[2]{\langle}{\rangle}{#1 \delimsize\vert #2}
%\definecolor{shadecolor}{named}{red!25!green!50!blue!75}
\colorlet{NextBlue}{red!25!green!50!blue!75}
%\colorlet{shadecolor}{orange!15}
\colorlet{shadecolor}{NextBlue!40}
\parindent 0in
\parskip 10pt
\geometry{margin=1in, headsep=0.25in}
\theoremstyle{definition}
\newtheorem{defn}{Definition}
\newtheorem{reg}{Rule}
\newtheorem{exer}{Exercise}
\newtheorem{note}{Note}
\begin{document}
\setcounter{section}{1}
\title{Klein-Gordon Equation}

%==============================================================
\pagestyle{fancy}
\fancyhf{}
\rhead{Physics 180}
\chead{Dirac Equation}
\lhead{Olyn D. Desabelle}
\rfoot{Page \thepage}
\setlength{\headheight}{12.0pt}

\begin{center}
{\LARGE \bf Dirac Equation}\\
%{\large Physics 170}\\
%Olyn D. Desabelle
\end{center}

\section*{Klein-Gordon Equation}%----------------------------------------

The Einstein energy-momentum relationship when expressed in terms of operators becomes:

\begin{align*}
    E^2 &= \mathbf{p}^2 + m^2\\
    \hat{E}^2\psi(\mathbf{x},t) &= \hat{\mathbf{p}}^2\psi(\mathbf{x},t) + m^2\psi(\mathbf{x},t)
\end{align*}

with the operators rewritten as $\hat{\mathbf{p}} = -i\nabla$ and $\hat{E} = i\frac{\partial}{\partial t}$ , the \textbf{Klein-Gordon wave equation} is given by:

\begin{align}
    \frac{\partial^2\psi}{\partial t^2} = \nabla^2\psi - m^2\psi
\end{align}

or in Lorentz-invariant form,

\begin{align}
    (\partial^{\mu}\partial_{\mu} + m^2)\psi = 0
\end{align}

where

\begin{align*}
    \partial^{\mu}\partial_{\mu} = \frac{\partial^2}{\partial t^2} - \frac{\partial^2}{\partial x^2} - \frac{\partial^2}{\partial y^2} - \frac{\partial^2}{\partial z^2}
\end{align*}

This has plane wave solutions in the form of:

\begin{align}
    \psi(\mathbf{x},t) = Ne^{i(\mathbf{p}\cdot\mathbf{x} - Et)}
\end{align}


\section*{The Dirac Equation}%----------------------------------------

Dirac sought to find a wave equation that was first order in space and time derivatives (KGE had seecond order):

\begin{align}
    \hat{E}\psi &= (\mathbf{\alpha}\cdot\hat{\mathbf{p}} + \beta m)\psi\\
    i \frac{\partial}{\partial t}\psi &= \left( -i\alpha_x \frac{\partial}{\partial x} -i\alpha_y \frac{\partial}{\partial y} -i\alpha_z \frac{\partial}{\partial z} + \beta m \right)\psi
\end{align}

in order to satisfy the KGE, $\alpha$ and $\beta$ must satisfy:

\begin{align*}
    \alpha_x^2 = \alpha_y^2 = \alpha_z^2 = \beta^2 &= I\\
    \alpha_j\beta + \beta\alpha_j &= 0\\
    \alpha_j \alpha_k + \alpha_k\alpha_j &= 0 \;\;\; (j\neq k)
\end{align*}

$\mathbf{\alpha}$ components and $\beta$ are usually explicitly written using the \textbf{Dirac-Pauli representation}:

\begin{align*}
    \beta = \begin{pmatrix}
        I & 0\\
        0 & -I
    \end{pmatrix}
    \;\;\; &
    \alpha_i = \begin{pmatrix}
        0 & \sigma_i\\
        \sigma_i & 0
    \end{pmatrix}\\
    I = \begin{pmatrix}
        1 & 0 \\
        0 & 1 \\
    \end{pmatrix}
    \;\;\;
    \sigma_x = \begin{pmatrix}
        0 & 1 \\
        1 & 0
    \end{pmatrix}
    \;\;\;&
    \sigma_y = \begin{pmatrix}
        0 & -i\\
        i & 0
    \end{pmatrix}
    \;\;\;
    \sigma_z = \begin{pmatrix}
        1 & 0\\
        0 & -1
    \end{pmatrix}
\end{align*}

\section*{Probability density and probability current }%----------------------------------------

Starting with the KGE and after exploiting the Hermitian property of the $\alpha$ and $\beta$ matrices, a continuity equation may be obtained:

\begin{align}
    \nabla \cdot (\psi^{\dagger}\alpha\psi) + \frac{\partial (\psi^{\dagger}\psi)}{\partial t} = 0
\end{align}

with $\psi^{\dagger} = (\psi_1^*,\psi_2^*,\psi_3^*,\psi_4^*)$. The probability density and probability current then take the forms of:

\begin{align*}
    \rho &= \psi^{\dagger}\psi = |\psi_1|^2 +|\psi_2|^2 +|\psi_3|^2 +|\psi_4|^2 \\
    \mathbf{j} &= \psi^{\dagger}\mathbf{\alpha}\psi
\end{align*}

\section*{Covariant form of the Dirac equation}%----------------------------------------

The covariant form of the Dirac equation is given by:

\begin{align}
    (i\gamma^{\mu}\partial_{\mu} - m)\psi = 0
\end{align}

the gamma matrices are as follow:

\begin{align*}
    \gamma^0 = \begin{pmatrix}
        1 & 0 & 0 & 0\\
        0 & 1 & 0 & 0\\
        0 & 0 & -1 & 0\\
        0 & 0 & 0 & -1
    \end{pmatrix}
    \;\;\;&
    \gamma^1 = \begin{pmatrix}
        0 & 0 & 0 & 1\\
        0 & 0 & 1 & 0\\
        0 & -1 & 0 & 0\\
        -1 & 0 & 0 & 0
    \end{pmatrix}\\
    \gamma^2 = \begin{pmatrix}
        0 & 0 & 0 & -i\\
        0 & 0 & i & 0\\
        0 & i & 0 & 0\\
        -i & 0 & 0 & 0
    \end{pmatrix}
    \;\;\;&
    \gamma^4 = \begin{pmatrix}
        0 & 0 & 1 & 0\\
        0 & 0 & 0 & -1\\
        -1 & 0 & 0 & 0\\
        0 & 1 & 0 & 0
    \end{pmatrix}\\
\end{align*}

the matrices satisfy conditions such as:

\begin{align*}
    (\gamma^0)^2 &= I\\
    (\gamma^k)^2 &= -I\\
    \gamma^{\mu}\gamma^{\nu} &= -\gamma^{\nu}\gamma^{\mu}\;\;\; (\mu\neq\nu)\\
    \{\gamma^{\mu},\gamma^{\nu}\} &\equiv  \gamma^{\mu}\gamma^{\nu} +  \gamma^{\nu}\gamma^{\mu} = 2g^{\mu\nu}\\
    \gamma^{0\dagger} &= \gamma^0\\
    \gamma^{k\dagger} &= -\gamma^k\\
\end{align*}

\section*{The adjoint spinor and covariant current}%----------------------------------------
%----------------------------------------

The adjoint spinor may be defined as:

\begin{align}
    \bar{\psi} &= \psi^{\dagger}\gamma^0 = (\psi_1^*,\psi_2^*,-\psi_3^*,-\psi_4^*)
\end{align}

the four-vector current may then be written as:

\begin{align}
    j^{\mu} = \bar{\psi}\gamma^{\mu}\psi
\end{align}

\section*{Solutions to Dirac equation}

Solutions may take the form of:

\begin{align}
    \psi(\mathbf{x},t) = u(E,\mathbf{p})e^{i(\mathbf{p}\cdot\mathbf{x}-Et)}
\end{align}

satisfying the Dirac  equation then the Dirac spinor $u$ satisfies:

\begin{align}
    (\gamma^{\mu}p_{\mu}-m)u = 0
\end{align}

\subsection*{Particles at rest}%-------------------------------

Particles at rest have $\mathbf{p}=0$, thus we have:

\begin{align}
    \psi &= u(E,0)e^{-iEt}\\
    E\gamma^0u = mu
\end{align}

this gives solutions of:

\begin{align*}
    \psi_1 = N\begin{pmatrix}
        1\\
        0\\
        0\\
        0
    \end{pmatrix} e^{-imt}
    \;\;\;&
    \psi_2 = N\begin{pmatrix}
        0\\
        1\\
        0\\
        0
    \end{pmatrix} e^{-imt}\\
    \psi_3 = N\begin{pmatrix}
        0\\
        0\\
        1\\
        0
    \end{pmatrix} e^{+imt}
    \;\;\;&
    \psi_1 = N\begin{pmatrix}
        0\\
        0\\
        0\\
        1
    \end{pmatrix} e^{+imt}\\
\end{align*}

\subsection*{General free-particle solutions}%-------------------------------

General free-particle solutions take the form of

\begin{align}
    \psi_i &= u(E,\mathbf{p})e^{i(\mathbf{p}\cdot\mathbf{x}-Et)}\\
\end{align}

where $u_i$ take form of:

\begin{align*}
    u_1 = N_1\begin{pmatrix}
        1\\
        0\\
        \frac{p_z}{E+m}\\
        \frac{p_x+ip_y}{E+m}
    \end{pmatrix}
    \;\;\; &
    u_2 = N_2\begin{pmatrix}
        0\\
        1\\
        \frac{p_x-ip_y}{E+m}\\
        \frac{-p_z}{E+m}
    \end{pmatrix}\\
    u_3 = N_3\begin{pmatrix}
        \frac{p_z}{E-m}\\
        \frac{p_x+ip_y}{E-m}\\
        1\\
        0
    \end{pmatrix}
    \;\;\; &
    u_4 = N_4\begin{pmatrix}
        \frac{p_x-ip_y}{E-m}\\
        \frac{-p_z}{E-m}
        0\\
        1\\
    \end{pmatrix}\\
\end{align*}

\subsection*{Antiparticle solutions}%-------------------------------

Antiparticle solutions take the form of

\begin{align}
    \psi_i &= v(E,\mathbf{p})e^{i(\mathbf{p}\cdot\mathbf{x}-Et)}\\
\end{align}

where $v_i$ take form of:

\begin{align*}
    v_1 = N\begin{pmatrix}
        \frac{p_x-ip_y}{E+m}\\
        \frac{-p_z}{E+m}\\
        0\\
        1\\
    \end{pmatrix}
    \;\;\; &
    v_2 = N\begin{pmatrix}
        \frac{p_z}{E+m}\\
        \frac{p_x+ip_y}{E+m}\\
        1\\
        0\\
    \end{pmatrix}\\
    v_3 = N\begin{pmatrix}
        1\\
        0\\
        \frac{p_z}{E-m}\\
        \frac{p_x+ip_y}{E-m}\\
    \end{pmatrix}
    \;\;\; &
    u_4 = N_4\begin{pmatrix}
        0\\
        1\\
        \frac{p_x-ip_y}{E-m}\\
        \frac{-p_z}{E-m}
    \end{pmatrix}\\
\end{align*}

Usually, it is more natural to work with positive energy solutions for particles and antiparticles ($\{u_1,u_2,v_1,v_2\}$)

\subsection*{Helicity}%----------------------------------------
%----------------------------------------

Particle helicity spinors take the form of:

\begin{align*}
    u_{\uparrow} = \sqrt{E+m}\begin{pmatrix}
        c\\
        se^{i\phi}\\
        \frac{p}{E+m}c\\
        \frac{p}{E+m}se^{i\phi}
    \end{pmatrix}
    \;\;\;&
    u_{\downarrow} = \sqrt{E+m}\begin{pmatrix}
        -s\\
        ce^{i\phi}\\
        \frac{p}{E+m}s\\
        -\frac{p}{E+m}ce^{i\phi}
    \end{pmatrix}\\
    u_{\uparrow} \approx \sqrt{E}\begin{pmatrix}
        c\\
        se^{i\phi}\\
        c\\
        se^{i\phi}
    \end{pmatrix}
    \;\;\;&
    u_{\downarrow} \approx \sqrt{E}\begin{pmatrix}
        -s\\
        ce^{i\phi}\\
        s\\
        -ce^{i\phi}
    \end{pmatrix}
    \;\;\; (E \gg m)
\end{align*}

Antiparticle helicity spinors take the form of:

\begin{align*}
    v_{\uparrow} = \sqrt{E+m}\begin{pmatrix}
        \frac{p}{E+m}s\\
        -\frac{p}{E+m}ce^{i\phi}\\
        -s\\
        ce^{i\phi}
    \end{pmatrix}
    \;\;\;&
    v_{\downarrow} = \sqrt{E+m}\begin{pmatrix}
        \frac{p}{E+m}c\\
        \frac{p}{E+m}se^{i\phi}\\
        c\\
        se^{i\phi}
    \end{pmatrix}\\
    v_{\uparrow} = \sqrt{E}\begin{pmatrix}
        s\\
        -ce^{i\phi}\\
        -s\\
        ce^{i\phi}
    \end{pmatrix}
    \;\;\;&
    v_{\downarrow} = \sqrt{E}\begin{pmatrix}
        c\\
        se^{i\phi}\\
        c\\
        se^{i\phi}
    \end{pmatrix}
    \;\;\; (E \gg m)
\end{align*}


%
%----------------------------------------
%---------------------------------------
%----------------------------------------
\end{document}