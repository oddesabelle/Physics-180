\documentclass[10pt]{article}
\usepackage[utf8]{inputenc}	% Para caracteres en español
\usepackage{amsmath,amsthm,amsfonts,amssymb,amscd}
\usepackage{multirow,booktabs}
\usepackage[table]{xcolor}
\usepackage{fullpage}
\usepackage{lastpage}
\usepackage{enumitem}
\usepackage{fancyhdr}
\usepackage{mathrsfs}
\usepackage{wrapfig}
\usepackage{setspace}
\usepackage{calc}
\usepackage{multicol}
\usepackage{cancel}
\usepackage[retainorgcmds]{IEEEtrantools}
\usepackage[margin=3cm]{geometry}
\usepackage{amsmath}
\newlength{\tabcont}
\setlength{\parindent}{0.0in}
\setlength{\parskip}{0.05in}
\usepackage{empheq}
\usepackage{framed}
\usepackage[most]{tcolorbox}
\usepackage{xcolor}
\usepackage{float}
\usepackage[english]{babel}
\usepackage[utf8]{inputenc}
\usepackage{graphicx}
\usepackage[colorinlistoftodos]{todonotes}
\usepackage[version=4]{mhchem}
\usepackage{physics}
\usepackage{mathtools}
%\DeclarePairedDelimiter\bra{\langle}{\rvert}
%\DeclarePairedDelimiter\ket{\lvert}{\rangle}
%\DeclarePairedDelimiterX\braket[2]{\langle}{\rangle}{#1 \delimsize\vert #2}
%\definecolor{shadecolor}{named}{red!25!green!50!blue!75}
\colorlet{NextBlue}{red!25!green!50!blue!75}
%\colorlet{shadecolor}{orange!15}
\colorlet{shadecolor}{NextBlue!40}
\parindent 0in
\parskip 10pt
\geometry{margin=1in, headsep=0.25in}
\theoremstyle{definition}
\newtheorem{defn}{Definition}
\newtheorem{reg}{Rule}
\newtheorem{exer}{Exercise}
\newtheorem{note}{Note}
\begin{document}
\setcounter{section}{1}
\title{Klein-Gordon Equation}

%==============================================================
\pagestyle{fancy}
\fancyhf{}
\rhead{Physics 180}
\chead{Dirac Equation}
\lhead{Olyn D. Desabelle}
\rfoot{Page \thepage}
\setlength{\headheight}{12.0pt}

\begin{center}
{\LARGE \bf Dirac Equation}\\
%{\large Physics 170}\\
%Olyn D. Desabelle
\end{center}

\section*{Klein-Gordon Equation}%----------------------------------------

The Einstein energy-momentum relationship when expressed in terms of operators becomes:

\begin{align*}
    E^2 &= \mathbf{p}^2 + m^2\\
    \hat{E}^2\psi(\mathbf{x},t) &= \hat{\mathbf{p}}^2\psi(\mathbf{x},t) + m^2\psi(\mathbf{x},t)
\end{align*}

with the operators rewritten as $\hat{\mathbf{p}} = -i\nabla$ and $\hat{E} = i\frac{\partial}{\partial t}$ , the \textbf{Klein-Gordon wave equation} is given by:

\begin{align}
    \frac{\partial^2\psi}{\partial t^2} = \nabla^2\psi - m^2\psi
\end{align}

or in Lorentz-invariant form,

\begin{align}
    (\partial^{\mu}\partial_{\mu} + m^2)\psi = 0
\end{align}

where

\begin{align*}
    \partial^{\mu}\partial_{\mu} = \frac{\partial^2}{\partial t^2} - \frac{\partial^2}{\partial x^2} - \frac{\partial^2}{\partial y^2} - \frac{\partial^2}{\partial z^2}
\end{align*}

This has plane wave solutions in the form of:

\begin{align}
    \psi(\mathbf{x},t) = Ne^{i(\mathbf{p}\cdot\mathbf{x} - Et)}
\end{align}


\section*{The Dirac Equation}

%\subsection*{Dilute Interacting Bose gas}
%----------------------------------------
%---------------------------------------
%----------------------------------------
\end{document}