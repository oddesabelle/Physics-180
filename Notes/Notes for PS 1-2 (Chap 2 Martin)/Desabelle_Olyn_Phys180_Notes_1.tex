\documentclass[10pt]{article}
\usepackage[utf8]{inputenc}	% Para caracteres en español
\usepackage{amsmath,amsthm,amsfonts,amssymb,amscd}
\usepackage{multirow,booktabs}
\usepackage[table]{xcolor}
\usepackage{fullpage}
\usepackage{lastpage}
\usepackage{enumitem}
\usepackage{fancyhdr}
\usepackage{mathrsfs}
\usepackage{wrapfig}
\usepackage{setspace}
\usepackage{calc}
\usepackage{multicol}
\usepackage{cancel}
\usepackage[retainorgcmds]{IEEEtrantools}
\usepackage[margin=3cm]{geometry}
\usepackage{amsmath}
\newlength{\tabcont}
\setlength{\parindent}{0.0in}
\setlength{\parskip}{0.05in}
\usepackage{empheq}
\usepackage{framed}
\usepackage[most]{tcolorbox}
\usepackage{xcolor}
\usepackage{float}
\usepackage[english]{babel}
\usepackage[utf8]{inputenc}
\usepackage{graphicx}
\usepackage[colorinlistoftodos]{todonotes}
\usepackage[version=4]{mhchem}
\usepackage{physics}
\usepackage{mathtools}
%\DeclarePairedDelimiter\bra{\langle}{\rvert}
%\DeclarePairedDelimiter\ket{\lvert}{\rangle}
%\DeclarePairedDelimiterX\braket[2]{\langle}{\rangle}{#1 \delimsize\vert #2}
%\definecolor{shadecolor}{named}{red!25!green!50!blue!75}
\colorlet{NextBlue}{red!25!green!50!blue!75}
%\colorlet{shadecolor}{orange!15}
\colorlet{shadecolor}{NextBlue!40}
\parindent 0in
\parskip 10pt
\geometry{margin=1in, headsep=0.25in}
\theoremstyle{definition}
\newtheorem{defn}{Definition}
\newtheorem{reg}{Rule}
\newtheorem{exer}{Exercise}
\newtheorem{note}{Note}
\begin{document}
\setcounter{section}{1}
\title{Nuclear phenomenology}

%==============================================================
\pagestyle{fancy}
\fancyhf{}
\rhead{Physics 180}
\chead{Nuclear phenomenology}
\lhead{Olyn D. Desabelle}
\rfoot{Page \thepage}
\setlength{\headheight}{12.0pt}

\begin{center}
{\LARGE \bf Nuclear phenomenology}\\
%{\large Physics 170}\\
%Olyn D. Desabelle
\end{center}

\section*{Nuclides}%----------------------------------------

%\subsection*{Dilute Interacting Bose gas}

Nuclides are typically written as:

$$\ce{^{A}_{Z}Y}$$

where we note that:

\begin{align*}
    &A\text{ is the }\textbf{mass/nucleon number}\text{ (\# of nucleons)}\\
    &Z\text{ is the }\textbf{proton/atomic number}\text{ (\# of protons)}\\
    &N\text{ is the }\textbf{neutron number}\text{ (\# of neutrons)}\\
\end{align*}

with $\mathbf{A = Z+N}$. We also note that isotopes with: same $A$ = isobars; same $Z$ = isotopes; same $N$ = isotones. Some elements have multiple isotopes, with different stabilty and abundance.
%\begin{shaded}
    %\textbf{Cooper pairs}\newline
%    \todo[inline, color = lime!20,caption={}]{
%        $$\ce{^{A}_{Z}Y}$$
        %$$\text{with } A = Z+N$$
%   }
    %where we note that:
    %\todo[inline, color=lime!20,caption={}]{
        %$A$ is the \textbf{mass/nucleon number} (\# of nucleons), \newline
        %$Z$ is the \textbf{proton/atomic number} (\# of protons), and \newline
        %$N$ is the \textbf{neutron number} (\# of neutrons).\newline
        %\begin{align}
        %&A\text{ is the }\textbf{mass/nucleon number}\text{ (\# of nucleons)}\\
        %&Z\text{ is the }\textbf{proton/atomic number}\text{ (\# of protons)}\\
        %&N\text{ is the }\textbf{neutron number}\text{ (\# of neutrons)}\\
%        \end{align}
%    }
%    with $\mathbf{A = Z+N}$. We also note that isotopes with: same $A$ = isobars; same $Z$ = isotopes; same $N$ = isotones. Some elements have multiple isotopes, with different stabilty and abundance.
%\end{shaded}

\section*{Nuclear shapes and sizes}%----------------------------------------

Nuclei may be treated as static charge distributions with normalization:

\begin{align}
    \int f(\mathbf{r})\text{d}^3\mathbf{r} = Ze
\end{align}

where $e$ is the electron charge. Under the Born approximation, the cross-section $\frac{d\sigma}{d\Omega}$ is given by:

\begin{align}
    \left(\frac{d\sigma}{d\Omega}\right)_0 &= 
    \frac{Z^2 \alpha^2 (\hbar c)^2}{4\beta^4 E^2 \sin^4(\theta/2)}
\end{align}

including the electron spin, the Mott cross-section is given by:

\begin{align}
    \left(\frac{d\sigma}{d\Omega}\right)_{\text{Mott}} &=
    \left(\frac{d\sigma}{d\Omega}\right)_0 [1-\beta^2\sin^2(\theta/2)]
\end{align}


In the nonrelativistic limit with no spin dependence, the Rutherford cross-section is given by:

\begin{align}
    \left(\frac{d\sigma}{d\Omega}\right)_{\text{Rutherford}} &=
    \frac{(\hbar c)^2 (\alpha Z)^2}{4m^2 v^4 \sin^4(\theta/2)}
\end{align}

Given $\mathbf{q} = \mathbf{p} - \mathbf{p}'$ where $\mathbf{p}$ and $\mathbf{p}'$ are initial and final electron momenta, the form factor $F(\mathbf{q}^2)$ (Fourier transform of charge distribution) is given by:

\begin{align}
    F(\mathbf{q}^2) &= \frac{1}{Ze} \int e^{i\mathbf{q}\cdot\mathbf{r}/\hbar} f(\mathbf{r})\; \text{d}^3\mathbf{r}\\
    F(\mathbf{q}^2) &= \frac{4\pi\hbar}{Zeq}\int_{0}^{\infty} r \rho(r) \sin\left(\frac{qr}{\hbar}\right)\;\text{d}r\\
    F(\mathbf{q}^2) &= \frac{4\pi}{Ze}\int_{0}^{\infty} f(r)r^2\;\text{d}r - \frac{4\pi\mathbf{q}^2}{6Ze\hbar^2}\int_0^{\infty}f(r) r^4\;\text{d}r + \dots\\
\end{align}

Experimental cross-section may be approximately related to the Mott cross-section by:

\begin{align}
    \left(\frac{d\sigma}{d\Omega}\right)_{\text{expt}} &=
    \left(\frac{d\sigma}{d\Omega}\right)_{\text{Mott}} |F(\mathbf{q}^2)|
\end{align}

Representing the nuclear charge distribution as a hard sphere where $\rho(r)$ is constant for $r\leq a$ and 0 otherwise, then the form factor simplifies to $F(\mathbf{q}^2)=3[\sin(b)-b\cos(b)]b^{-3}$.\newline

The charge distribution may be obtained from the form factor using:

\begin{align}
    f(\mathbf{r}) = \frac{Ze}{(2\pi)^3}\int F(\mathbf{q}^2) e^{-i\mathbf{q}\cdot\mathbf{r}/\hbar}\text{d}^3\mathbf{q}
\end{align}

which can be simplified into:

\begin{align}
    f(r) = \rho_{\text{ch}}(r) = \frac{\rho^0_{\text{ch}}}{1+e^{(r-a)/b}}
\end{align}

where $a\approx 1.07A^{1/3}\;\text{fm}$ and $b\approx 0.54\;\text{fm}$.\newline

Another important quantity would be the mean square charge radius:

\begin{align}
    \expval{r^2} &= \frac{1}{Ze} \int r^2 f(\mathbf{r})\;\text{d}^3\mathbf{r}\\
    \expval{r^2} &= -6\hbar^2\frac{\text{d}F(\mathbf{q}^2)}{\text{d}\mathbf{q}^2}\biggr|_{\mathbf{q}^2=0}
\end{align}

for very small values of $\mathbf{q}^2$, for medium and heavy nuclei, $\expval{r^2}^{1/2} = 0.94A^{1/3}\;\text{fm}$.



\section*{Semi-empirical mass formula (SEMF)}%----------------------------------------

The forces that bind nucleons also contribute to the total mass $M(Z,A)$ of an atom aside from the contributions of the proton (with mass $M_p$), neutron (with mass $M_n$), and electron (with mass $m_e$). This mass deficit is given by:

\begin{align}
    \Delta M(Z,A) = M(Z,A) - Z(M_p+m_e) - NM_n
\end{align}

multiplying each side of the equation by $-c^2$ gives the \textbf{binding energy} $B$. \newline

The atomic mass $M(Z,A)$ is given by a sum of terms:

\begin{align}
    M(Z,A) = \sum_{i=0}^{5} f_i(Z,A)
\end{align}

where each term is given by:

\begin{align*}
    \text{(mass of nucleons \& electrons) }\; f_0(Z,A) &= Z(M_p + m_e) + (A-Z)M_n\\
    \text{(volume term) }\; f_1(Z,A) &= -a_1 A\\
    \text{(surface term) }\; f_2(Z,A) &= a_2 A^{2/3}\\
    \text{(Coulomb term) }\; f_3(Z,A) &= a_3 \frac{Z(Z-1)}{A^{1/3}} \approx a_3 \frac{Z^2}{A^{1/3}}\\
    \text{(asymmetry term) }\; f_4(Z,A) &= a_4 \frac{(Z-A/2)^2}{A}\\
    \text{(pairing term) }\; f_5(Z,A) &= -a_5 A^{-1/2}\; \text{if $Z$ and $N$ are even}\\
    &= a_5 A^{-1/2}\; \text{if $Z$ and $N$ are odd}\\
    &= 0 \; \text{otherwise}\\
\end{align*}

the constants of each term have the respective values:

\begin{align*}
    a_1 &= a_v = 15.56 \; \text{MeV/c}^2\\
    a_2 &= a_s = 17.23 \; \text{MeV/c}^2\\
    a_3 &= a_c = 0.697 \; \text{MeV/c}^2\\
    a_4 &= a_a = 93.14 \; \text{MeV/c}^2\\
    a_5 &= a_p = 12 \; \text{MeV/c}^2\\
\end{align*}

%\section*{$\alpha$ emissions}%----------------------------------------
%\section*{Radioactive decay}%----------------------------------------
%---------------------------------------
%----------------------------------------
\end{document}