\documentclass[10pt]{article}
\usepackage[utf8]{inputenc}	% Para caracteres en español
\usepackage{amsmath,amsthm,amsfonts,amssymb,amscd}
\usepackage{multirow,booktabs}
\usepackage[table]{xcolor}
\usepackage{fullpage}
\usepackage{lastpage}
\usepackage{enumitem}
\usepackage{fancyhdr}
\usepackage{mathrsfs}
\usepackage{wrapfig}
\usepackage{setspace}
\usepackage{calc}
\usepackage{multicol}
\usepackage{cancel}
\usepackage[retainorgcmds]{IEEEtrantools}
\usepackage[margin=3cm]{geometry}
\usepackage{amsmath}
\newlength{\tabcont}
\setlength{\parindent}{0.0in}
\setlength{\parskip}{0.05in}
\usepackage{empheq}
\usepackage{framed}
\usepackage[most]{tcolorbox}
\usepackage{xcolor}
\usepackage{float}
\usepackage[english]{babel}
\usepackage[utf8]{inputenc}
\usepackage{graphicx}
\usepackage[colorinlistoftodos]{todonotes}
\usepackage[version=4]{mhchem}
\usepackage{physics}
\usepackage{mathtools}
%\DeclarePairedDelimiter\bra{\langle}{\rvert}
%\DeclarePairedDelimiter\ket{\lvert}{\rangle}
%\DeclarePairedDelimiterX\braket[2]{\langle}{\rangle}{#1 \delimsize\vert #2}
%\definecolor{shadecolor}{named}{red!25!green!50!blue!75}
\colorlet{NextBlue}{red!25!green!50!blue!75}
%\colorlet{shadecolor}{orange!15}
\colorlet{shadecolor}{NextBlue!40}
\parindent 0in
\parskip 10pt
\geometry{margin=1in, headsep=0.25in}
\theoremstyle{definition}
\newtheorem{defn}{Definition}
\newtheorem{reg}{Rule}
\newtheorem{exer}{Exercise}
\newtheorem{note}{Note}
\begin{document}
\setcounter{section}{1}
\title{Nuclear phenomenology}

%==============================================================
\pagestyle{fancy}
\fancyhf{}
\rhead{Physics 180}
\chead{Nuclear phenomenology}
\lhead{Olyn D. Desabelle}
\rfoot{Page \thepage}
\setlength{\headheight}{12.0pt}

\begin{center}
{\LARGE \bf Nuclear phenomenology}\\
%{\large Physics 170}\\
%Olyn D. Desabelle
\end{center}

\section*{Nuclides}%----------------------------------------

%\subsection*{Dilute Interacting Bose gas}

Nuclides are typically written as:

$$\ce{^{A}_{Z}Y}$$

where we note that:

\begin{align*}
    &A\text{ is the }\textbf{mass/nucleon number}\text{ (\# of nucleons)}\\
    &Z\text{ is the }\textbf{proton/atomic number}\text{ (\# of protons)}\\
    &N\text{ is the }\textbf{neutron number}\text{ (\# of neutrons)}\\
\end{align*}

with $\mathbf{A = Z+N}$. We also note that isotopes with: same $A$ = isobars; same $Z$ = isotopes; same $N$ = isotones. Some elements have multiple isotopes, with different stabilty and abundance.
%\begin{shaded}
    %\textbf{Cooper pairs}\newline
%    \todo[inline, color = lime!20,caption={}]{
%        $$\ce{^{A}_{Z}Y}$$
        %$$\text{with } A = Z+N$$
%   }
    %where we note that:
    %\todo[inline, color=lime!20,caption={}]{
        %$A$ is the \textbf{mass/nucleon number} (\# of nucleons), \newline
        %$Z$ is the \textbf{proton/atomic number} (\# of protons), and \newline
        %$N$ is the \textbf{neutron number} (\# of neutrons).\newline
        %\begin{align*}
        %&A\text{ is the }\textbf{mass/nucleon number}\text{ (\# of nucleons)}\\
        %&Z\text{ is the }\textbf{proton/atomic number}\text{ (\# of protons)}\\
        %&N\text{ is the }\textbf{neutron number}\text{ (\# of neutrons)}\\
%        \end{align*}
%    }
%    with $\mathbf{A = Z+N}$. We also note that isotopes with: same $A$ = isobars; same $Z$ = isotopes; same $N$ = isotones. Some elements have multiple isotopes, with different stabilty and abundance.
%\end{shaded}

\section*{Nuclear shapes and sizes}%----------------------------------------

Nuclei may be treated as static charge distributions with normalization:

\begin{align*}
    \int f(\mathbf{r})\text{d}^3\mathbf{r} = Ze
\end{align*}

where $e$ is the electron charge. Under the Born approximation, the cross-section $\frac{d\sigma}{d\Omega}$ is given by:

\begin{align*}
    \left(\frac{d\sigma}{d\Omega}\right)_0 &= 
    \frac{Z^2 \alpha^2 (\hbar c)^2}{4\beta^4 E^2 \sin^4(\theta/2)}
\end{align*}

including the electron spin, the Mott cross-section is given by:

\begin{align*}
    \left(\frac{d\sigma}{d\Omega}\right)_{\text{Mott}} &=
    \left(\frac{d\sigma}{d\Omega}\right)_0 [1-\beta^2\sin^2(\theta/2)]
\end{align*}

in the nonrelativistic limit with no spin dependence, the Rutherford cross-section is given by:

\begin{align*}
    \left(\frac{d\sigma}{d\Omega}\right)_{\text{Rutherford}} &=
    \frac{(\hbar c)^2 (\alpha Z)^2}{4m^2 v^4 \sin^4(\theta/2)}
\end{align*}


\section*{SEMF}%----------------------------------------


\section*{$\alpha$ emissions}%----------------------------------------
%---------------------------------------
%----------------------------------------
\end{document}